\section{Related Work}

The main purpose of our work is to help drivers reduce their fuel consumption.
Another way of reducing the stop-and-go behavior at traffic lights than influencing the drivers, is to adjust the phases of the traffic lights to the currect status of the traffic. One such mehtod has been proposed in \cite{SOTL} that generates green waves based on interactions with cars and traffic lights. Their results show a decrease in waiting time by 50 \% in their comparisons.

The authors of \cite{ITLC} also propose a mehtod for adjusting the traffic lights to the traffic. They use reinforcement learning and a voting system amongst cars at or near a traffic light to calculate the total gain of different light settings. Their results showed that they where able to reduce both congestion and waiting time.

The authors of \cite{VANETsim} have proposed an idea similar to ours, only they also rely on communication between nearby cars.
Similarly they use the simulator SUMO (Simulation of Urban MObility) and the interface TraCI (Traffic Control Interface) to simulate their methods. Their network is, however, very simple, and do not include any crossing traffic. Their results are therefore based on a very sterilised environment.

Another related issue is to re-route drivers in order to avoid congestion. This way, drivers, will be able to drive more smoothly an hence reduce their fuel consumption. 
Three strategies for re-routing drivers to avoid congestion has been proposed in \cite{congestionAvoidance}. They collect real time data on the congestion levels from both the cars and road-side sensors, and provide individual re-routing strategies to the drivers based on the results. Through simulations they show promissing results.


