\section{Model}
In this section we describe the model that is used.

\begin{comment}\\\\
\noindent\begin{tabular}{ll}
$\Sigma$ & alphabet\\
$\mathbb{R}$ & set of real numbers\\
$\mathbb{N}$ & set of natual numbers not including zero\\
$\mathbb{N}_0$ & set of natual numbers including zero
\end{tabular}
\end{comment}

\subsection{Map}
A map is a fully connected, directed graph, $G = (V, E)$ where $V$ is a set of vertices or junctions and $E$ is a set of edges representing roads.

Every edge $e\in E$ has a starting point $e.v_1$, an end point $e.v_2$, a length $e.l$ and a maximum speed $e.s$.
We limit the map such that every vertice has at least one egde and at most four egdes.

\begin{comment}
A map is fully connected, directed graph of vertices and edges, where edges, $E$, are annotered with a length, $l$, a maximum speed limit, $s$. The edge is directed from $v_1$ to $v_2$.
\begin{align}
\mathcal{V} &= \{v \mid v \in \Sigma^*\}\\
\mathcal{E} &= \{(v_1, v_2, l, s) \mid v_1, v_2 \in \mathcal{V}, l\in \mathbb{R}, s\in\mathbb{N}\}\\
\end{align}

With this, we can define a map as a pair of vertices and egdes.
\[
m = (V, E) \mid V\subseteq \mathcal{V}, E\subseteq \mathcal{E}
\]


We assume that
\begin{itemize}
\item Every edge has a length, maximum speed and at least one lane
\item Every vertice has at most four edges
\item Every vertice is has at least one edge
\item There is at least one connection for each egde
\item Egdes in connections are connected in $E$
\end{itemize}
\end{comment}

\subsection{Trajectory}
A trajectory, $T$, is a sequence of connected edges on the map, $\langle e_1, e_2, \dots e_n \rangle$, where $e_i.v_2 = e_{i+1}.v_1$ for $0\leq i< n$.
The vehicle starts at $e_1$ and moves along the sequence until $e_n$ has been reached.

\begin{comment}
A trajectory is a sequence of connected edges on the map, on which a vehicle can move. The vehicle starts at the first edge and moves along the sequence until the end of the sequence has been reached.
Let 
\[
\mathcal{T}_E = \{(e_1, e_2, \dots, e_n)\mid e_i \in E\}
\]
and be the set of all trajectories and $t\in \mathcal{T}$ be one such trajectory.
\end{comment}

\subsection{Vehicle}
A vehicle, $h$, is an entity that moves along a trajectory.
Each vehicle has a maximum speed, $h.s$, a constant acceleration, $h.a$ and deceleration, $h.d$.
We operate with two types of vehicles: cars and trucks. See Table \ref{table.vehicleTypes} for their specifications.
\begin{table}
\centering
\begin{tabular}{|l|l|l|}\hline
				& Car	& Truck \\\hline
Max speed 		& 		& \\\hline
Acceleration 	&		& \\\hline
Deceleration 	&		& \\\hline
\end{tabular}
\caption{Types of vehicles}\label{table.vehicleTypes}
\end{table}

\begin{comment}
A vehicle is an entity that moves along a trajectory. Each vehicle has a maximum speed and a constant acceleration and deceleration.
Let
\[
\mathcal{H} = \{(s, a, d)\mid s, a, d\in \mathbb{N}\}
\]
and be the set of all vehicles and $h\in \mathcal{H}$ be one such vehicle.
\end{comment}

\subsection{Traffic Light Phases}
The phases of a traffic light, $l.p$ is defined as $\langle(g_1, t_1),(g_2, t_2),\dots, (g_n, t_n) \rangle$ where the light setting is $g_i\in \{r, y, g\}$ for $t_i$ seconds.
The phases of a traffic light defines the light settings when possible sensors are not triggered. 
\begin{comment}
A traffic light controls the flow of traffic at a junction (vertice). 
There is a program of red, yellow and green light setting for each connection between two lanes in the junction.
We call this a program or \textit{phase}. A phase, $P$, is a sequence of light settings and time spans. A light setting can either be green, red or yellow. At a green light traffic is allowed to pass the junction along the connection, at red light, traffic is stopped, and a yellow light setting signals a change between red and green. Hence $P$ is defined as
\[
\mathcal{P} =\{ \langle(g_1, t_1),(g_2, t_2),\dots, (g_n, t_n) \rangle\mid g_i\in \{r, y, g\}, t_i\in \mathbb{N}\}
\]
\end{comment}

\subsection{Sensors}
A sensor can detect whether a vehicle is located at the sensor or not. Sensors can be induction loops, video cameras or others. We let $\mathcal{S}$ be the set of sensors.

\subsection{Connections}
A road segment can have more than one lane, and we therefore need to define the connections between all lanes in the traffic lights.
These connections represent the two lanes one can travel between in a junction and the traffic light phase for that connection.
A connection, $c$ is bidirectional between two egdes, $c.e_1$ and $c.e_2$ and their associated lane identifier, $c.l_1$ and $c.l_2$. A connection is also associated with the vertice, $c.v$ that connects the two egdes and the phase, $c.p$.

\begin{comment}
In addition to egdes and vertices, we define et set of connections between two egdes. These connections represent the two lanes one can travel between in a junction. Each connection has a phase of traffic light settings.Lanes are indexed from $0$, and the set of connections is defined as
\[
\mathcal{C}_{(V, E)} = \{(e_1, l_1, e_2, l_2, p) \mid e_1, e_2 \in E, l_1, l_2\in \mathbb{N}_0, p\in \mathcal{P}\}
\]
We assume that $e_1$ and $e_2$ are connected in $(V, E)$.
\end{comment}

\subsection{Traffic Light}
A traffic light, $l$ controls the flow of traffic at a junction (vertice), and is associated with that vertice, $l.v$ and a set of sensors, $l.s$.

\begin{comment}
Now, we can describe the set of traffic lights as a triple of a vertice, a set of connections with a phase and a set of sensors.
\[
\mathcal{L}_{(V,E)} = \{(v, c, s)\mid v\in V, c\subseteq \mathcal{C}_{(V,E)}, s\in \mathcal{S}\}
\]
\end{comment}





