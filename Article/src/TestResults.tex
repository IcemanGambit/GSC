\section{Test Results}\label{sec:Test}
In this section, we test the proposed techinque on the use case explained in the previouse section.

% Verfication
% - Distance
% - Speed
% Fuel
% Time
% Congestion


\begin{table}
\centering
\begin{tabular}{|l|l|cc|cc|}\hline
Percent using 			& With/w.o. 	& \multicolumn{2}{c|}{Fuel} 	& \multicolumn{2}{c|}{Time}\\
\tech					&\tech					& $ml$	& Diff.		&	$s$	& Diff.\\\hline
\multirow{1}{*}{0\%}	& Without	&	150 &	0\%	&	108 & 0\%		\\\hline
\multirow{2}{*}{10\%}	& With 	&	100 &	20\%	&	108 & 0\%		\\
						& Without 	&	151 &	-1\%	&	108 & 0\%		\\\hline
\multirow{2}{*}{50\%}	& With	&	100	&	\%		&	108 & 0\%\\
						& Without	&	150	&	\%		&	108 & 0\%\\\hline
\multirow{1}{*}{100\%}	& With	&	100	&	\%		&	108 & 0\%\\\hline
\end{tabular}
\caption{Average values with out sensors. Diff. is the difference compared to 0 \% using \tech}
\label{tb:TestResults:total}
\end{table}

\subsection{Verification of Results}
It is imporant to verify that the results obtained from the simulator actually matches the real-world observations.
We compare the results from the simulator with GPS trajectories collected in a project called Spar P\aa Farten\cite{}.
If the results seem similar, we may deduce that the simulator accurately emulates the real world.
We compare the simulator with the GPS data on two fronts: travel distance and driving speed.
We only look at data from the main road of Hobro in the northern direction.

Travel distance as a function over time is plotted in Figure~\ref{fig:TestResults:realDistance}. 
We see that the curves flatens at about 300 m, 600 m, 900 m and 1200 m, indicating that the vehicles are stationary.
This corresponds well with the dimentions of Hobrovej and match the four regulated traffic lights.
\begin{figure}[htb]
\includegraphics[width=0.45\textwidth]{../images/Real/RealDistance.png}
\caption{Travel distance for real GPS trajectories}
\label{fig:TestResults:realDistance}
\end{figure}

Figure~\ref{fig:TestResults:distance0} also plots the travel distance as a function over time, only for the results of the simulator.
We see that the simulation resembles the real-world results quite well, however, the simulated vehicles tend to hold longer at the traffic lights. 
This is because the traffic light phases of the simulation are too different compared to real traffic lights. 
Besides that, the acceleration profiles look similar, and the map has the correct dimensions.

\begin{figure}[htb]
\includegraphics[width=0.45\textwidth]{../images/tp0c1_0/distanceUncontrolled0.png}
\caption{Travel distance for simulated data without \tech}
\label{fig:TestResults:distance0}
\end{figure}

The driving speed as a function over time for the GPS trajectories is plotted in Figure~\ref{fig:TestResults:RealSpeed}.
In order to read the data, only 20 trajectories have been plotted in this graph.
First of all, we see that at most drivers drive at around 50 km/h, and not the speed limit of 60 km/h. 
Some also drive much faster than that.
Besides that, we see that many drop to around 0 km/h and that some are able to avoid a full stop.

\begin{figure}[htb]
\includegraphics[width=0.45\textwidth]{../images/Real/RealSpeed.png}
\caption{Speed for real GPS trajectories}
\label{fig:TestResults:RealSpeed}
\end{figure}

Plots of the driving speed for the simulated vehicles can be seen in Figure~\ref{fig:TestResults:speed0} as a function over time.
The behaviour of these simulated vehicles are much different from the real drivers.
The simulated drivers almost always either accelerate to the maximum speed or decelerate to 0 $km/h$. 
A few decelerates to 40 $km/h$, but that is most likely because the traffic light turns green just before the arrive.
No one neither breaks the speed limit.
The simulated vehicles therefore have a much more aggresive driving behaviour than what we see in the real world data.
This will also mean that the use more fuel than a real driver would use.

%TODO: Can we get 20 random plots in stead of the first 20?
\begin{figure}[htb]
\includegraphics[width=0.45\textwidth]{../images/tp0c1_0/speedUncontrolled0.png}
\caption{Speed for simulated data without \tech}
\label{fig:TestResults:speed0}
\end{figure}

\subsection{Fuel Consumption}
The purpose of \tech is to reduce the fuel consumption at least for the vehicles using the system. 
We use SUMO's build-in function to calculate the vehicles fuel consumption, which is explained in \cite{SUMOFuel}.

Figure~\ref{fig:TestResults:fuelTotal} plots the fuel consumption for all vehicles in the network for two different simulations: one where no vehicles use \tech (blue) and one where all vehicles are using \tech (green).
The amount of vehicles are the same in both simulations, and they drive the same routes with the same departure times. 
The only difference is whether they use \tech or not.

The average fuel consumption when all vehicles use \tech is about $90 ml$ and about $130 ml$ when no one use \tech.
We therefore see a significant overall reduction in fuel consumption on about 20 \% in this setting if \tech is used by everybody.
Looking at just the selected route through the network, we see the same tendency (see Figure~\ref{fig:TestResults:fuelRoute}). 
The average fuel consumption with \tech is about $114 ml$ and about $166 ml$ without, resulting in a reduction at 31.3 \%.
%TODO: Consider using 10% in stead of 100% in the two figures below.
\begin{figure}[htb]
\includegraphics[width=0.45\textwidth]{../images/tp0c1_0/fuelTotal.png}
\caption{Fuel consumption for all vehicles in the network. Blue: 100\% using \tech, green: 0\% using \tech}
\label{fig:TestResults:fuelTotal}
\end{figure}

\begin{figure}[htb]
\includegraphics[width=0.45\textwidth]{../images/tp0c1_0/fuelRoute.png}
\caption{Fuel consumption for all vehicles on the selected route}
\label{fig:TestResults:fuelRoute}
\end{figure}

It is, however, more interesting to investigate the influence \tech has when only a few drivers use it.
Figure~\ref{fig:TestResults:combinedFuel} shows the average fuel consumption for four different penetration rates (0 \%, 10 \%, 50 \% and 100 \%) on the main route.
The blue columns show the average fuel consumption for those not using \tech, and the green columns show the average fuel consumption for those using \tech.
The left- right-most columns hence show the same average levels as in Figure~\ref{fig:TestResults:fuelRoute}.
When \tech is used in 10 \% of the vehicles, we see a small decrease in fuel consumption for those not using it, and a significant reduction for those using \tech.
The reduction is about 20 \% (from 158 $ml$ to 120 $ml$).
When the penetration rate increases to 50 \%, we see that the fuel consumption decreases slightly for those not using \tech, and increases slightly for those using it.
This is most likely because vehicles start platooning (driving in groups), where vehicles using \tech force other vehicles to drive slower and as the recommended speed may match the traffic lights less when the vehicles huddle together.
The average fuel consumption when all vehicles use \tech, is about 130 $ml$, which 10 \% more than the average of the 10 \% using \tech, but still 20 \% lower than no one using \tech.

\begin{figure}[htb]
\includegraphics[width=0.45\textwidth]{../images/tp0c1_0/combinedFuel.png}
\caption{Fuel consumption for different penetration rates on the main route}
\label{fig:TestResults:combinedFuel}
\end{figure}

\begin{comment}
It is, however, more interesting to investigate the influence \tech has when only a few drivers use it.
Figure~\ref{fig:TestResults:fuelControlled10} and~\ref{fig:TestResults:fuelUncontrolled10} plot the fuel consumption when 10\% of the vehicles are using \tech - Figure~\ref{fig:TestResults:fuelControlled10} shows the 10 \% using \tech, and Figure~\ref{fig:TestResults:fuelUncontrolled10} show the 90 \% not using \tech.
We see that the average fuel consumption is about $119 ml$ for the 10 \% using \tech, and about $168$ for the 90 \% not using it.
The fuel consumption is signifinatly less for those who use \tech, and the fuel consumption is only increased by $2 ml$ for those who do not use \tech compared to the situation where no vehicles used \tech.
We see the same results when 50 \% of the vehicles use \tech.
\begin{figure}[h]
\includegraphics[width=0.5\textwidth]{../images/tp0c1_0/fuelRouteControlled10.png}
\caption{Fuel consumption for the 10 \% of the vehicles driving with \tech on the selected route}
\label{fig:TestResults:fuelControlled10}
\end{figure}
\begin{figure}[h]
\includegraphics[width=0.5\textwidth]{../images/tp0c1_0/fuelRouteUncontrolled10.png}
\caption{Fuel consumption for the 90 \% of the vehicles driving without \tech on the selected route}
\label{fig:TestResults:fuelUncontrolled10}
\end{figure}
\end{comment}

We see this fuel reduction is because the vehicles accelerate less rapidly. 
Figure~\ref{fig:TestResults:distance100} shows the travel distance as a function over time when all vehicles use \tech, and Figure~\ref{fig:TestResults:speed100} plots the driving speed as a function over time.
We see that these curves are much more smooth, because they avoid stopping, and hence has to accelerate less.
Some vehicles still have to stop, partly because it is impossible to drive the distance between the traffic lights in the amount of time left before it turns green, and partly because there are other vehicles blocking their path.
\begin{figure}[htb]
\includegraphics[width=0.45\textwidth]{../images/tp0c1_0/distanceControlled100.png}
\caption{Travel distance for simulated data when 100\% use \tech}
\label{fig:TestResults:distance100}
\end{figure}

\begin{figure}[htb]
\includegraphics[width=0.45\textwidth]{../images/tp0c1_0/speedControlled100.png}
\caption{Driving speed for simulated data when 100\% use \tech}
\label{fig:TestResults:speed100}
\end{figure}

We can therefore conclude that in the tested use case, we see a significate reduction in fuel consumption even when it is only implemented in a small subset of the vehicles. 
Moreover, vehicles using \tech does not have a negative influence the other fuel consumption of the vehicles.

\subsection{Travel time}
Reducing the fuel consumption is important, but it cannot be at the expense of traffic flow or the vehicles travel time.
Figure~\ref{fig:TestResults:combinedTime} show the average travel time for all vehicles on all routes in the network at different levels of penetration.
With 0 \% using \tech we see an average travel time of $150$ seconds. 
Introducing the systemt to 10 \% of the vehicles does not change this significantly. 
However, as the penetration rate increase, we see a small reduction in travel time. 
This is mostly due to the fact that vehicles do not make a full stop at traffic lights, and therefore are able to start faster.

\begin{figure}[htb]
\includegraphics[width=0.45\textwidth]{../images/tp0c1_0/combinedTime.png}
\caption{Travel time for all routes at different level of vehicles using \tech}
\label{fig:TestResults:combinedTime}
\end{figure}

\subsection{Congestion levels}
Normal traffic usualy have peak periods where the traffic is higher than normal \cite{Vejdir}. 
In order to test the effect of the system with different congestion levels we repeated the simulation while changing the rate with which vehicles are spwaning. 
The fuel consumption at different penetration rates and congestion levels can be seen in Figure \ref{fig:TestResults:congestionFuel}. 
It is clear that the effect of the system decreases when the congestion level increases. 
Traffic moves slower through the network and groups of blocking vehicles build up at the traffic lights. 
This reduce the effect of the system as it can no longer predict an accurate distance to the traffic light.
\begin{figure}[htb]
\includegraphics[width=0.45\textwidth]{../images/fuelCongestion.png}
\caption{Fuel consumption at different penetration and congestion levels}
\label{fig:TestResults:congestionFuel}
\end{figure}

In Figure \ref{fig:TestResults:congestionTime} we see the average time of all vehicles driving in the simulation compared to the diffrent congestion levels. 
With low congestion there is almost no difference in the travel time.
With the high congestion levels the system shows a positive effect with 50 \% or more users. 
In a congested network large queues of vehicles build up. 
A large portion of the time a traffic light is green is wasted on the time it takes to get the queue moving. 
A queue of vehicles accelerrating spend less time gaining speed if they are already moving slowly. 
This is likly the reasion that the average time of all vehicles improve if enough are using the system.
\begin{figure}[htb]
\includegraphics[width=0.45\textwidth]{../images/timeCongestion.png}
\caption{Travel time at different penetration and congestion levels}
\label{fig:TestResults:congestionTime}
\end{figure}



