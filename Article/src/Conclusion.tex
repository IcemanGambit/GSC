\section{Conclusion}\label{sec:Conclusion}

In this paper we have proposed a technique, \tech that use simple calculations reduce fuel consumption and improve traffic flow for vehicles.
\tech relies on access to traffic light phases but does not require further communication, e.g. with other vehicles.
The system is evaluated through simulations based on a 1.6 $km$ stretch of a larger road in Aalborg, Denmark with 4 pretimed traffic lights and crossing traffic.
Routes and congestion levels were estimated based on GPS trajectories, and the traffic light settings were based on \cite{vejtrafik}. %TODO: Too much?
The results of the simulations is compared with GPS trajectories and found to be a useful approximation.
The simulations show a 30 \% average reduction in fuel consumption for vehicles using \tech both when 10 \%, 50\% and 100\% of the vehicles use the system.
The fuel consumption of the other vehicles is not significantly effected.
Traffic flow is measured as the average travel time of all vehicles in the network, in that a decrease in travel time indicates an increase in traffic flow.
Tests find that the the average travel time decreases as the penetration rate increases. 
We also find that if all vehicles use the system, then more vehicles can pass though the network as the congestion level increases. %TODO: Not sure this is formulated correctly!

\section{Acknowledgements}
We will like to give our thanks to Harry Lahrmann for helping us understande the problems with traffic.
We also like to thank Niels Agerholm and Niels Ulrich Clausen from the city of Aalborg for providing current traffic light settings.
Thanks to Aalborg Univerisity and especially Benjamin Bjerre Krogh how provided GPS trajectories such that we validate our results.




