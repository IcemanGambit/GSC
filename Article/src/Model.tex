\section{Model}
In this section we describe the model that is used to find and test the recommended speed.
See Figure~\ref{fig:Introduction:network} for references.

\subsection{Map}
A map is a directed graph, $G = (V, E)$ where $V$ is a set of vertices representing road junctions and $E$ is a set of edges representing road segments.
Every edge $\edge\in E$ has a starting point \estart, an end point \eend and a maximum speed \espeed. 
The lenght of an edge is the euclidean distance between $\estart$ and \eend denoted as \elength.
For simplicity we limit the map such that every vertice is connected to the network through by one or more edges.
In figure~\ref{fig:Introduction:network} we have five egdes and one vertex, as the road coming from north is unidirectional.

\subsection{Lanes}
Every edge have a number of lanes $\elane \subseteq \heltal$, and each lane has an unique identifier for that egde as shown in Figure \ref{fig:Introduction:network}. 
A vehicle can at any time jump from one lane, $\elane_i$, to an adjacent lane, $\elane_j$ where $j=i\pm1, j\geq 0$

\subsection{Traffic Light Phases}
SUMO limits the allowed light settings to $red$, $yellow$ and $green$.
A phase of a traffic light, $\phase$ is defined as $\langle(\light_1, \ti_1),(\light_2, \ti_2),\dots, (\light_n, \ti_n) \rangle$ where the light setting is $\light_i\in \{red, yellow, green\}$ for $\ti_i$ seconds.
The phases of a traffic light defines the light settings when possible sensors are not triggered.

\subsection{Connections}
Some traffic lights do not always show the same signal to vehicles going in diffrent directions.
We model this by having \textit{connections} in the intersection. 
Each possible legal pass through the intersection is a connection as shown in the \textit{connection table} in Table \ref{tab:Introduction:connectionTable}. 
A connection, $\con$ is hence unidirectional from one egde, \cestart to another egde, \ceend at their associated lane identifier, \clstart and \clend. 
A connection is also associated with the vertice, $\cver$ that connects the two egdes and the phase, $\cphase$.

\begin{table}[h]
\centering
\begin{tabular}{|l|l|}
\hline
from & to \\ \hline
1 & 7 \\ \hline
4 & 3 \\ \hline
4 & 7 \\ \hline
5 & 3 \\ \hline
6 & 2 \\ \hline
\end{tabular}
\caption{Connection table of the network shown in Figure \ref{fig:Introduction:network} }
\label{tab:Introduction:connectionTable}
\end{table}

\begin{comment}
\subsection{Trajectory}
A trajectory, $T$, is a sequence of connected edges on the map, $\langle e_1, e_2, \dots e_n \rangle$, where $e_i.v_2 = e_{i+1}.v_1$ for $0\leq i< n$.
The vehicle starts at $e_1$ and moves along the sequence until $e_n$ has been reached.
\end{comment}

\subsection{Routes}
A route, $\route$, is a sequence of edges on the map, $\langle \edge_1, \edge_2, \dots \edge_n \rangle$, where $\eendi{i} = \estarti{i+1}$ for $0\leq i< n$.
The vehicle starts at $\edge_1$ and moves along the sequence until $\edge_n$ has been reached.

\subsection{Vehicles}
A vehicle, $\veh$, is an entity that moves along a route.
Each vehicle has an identifier, $\vehid$ a maximum speed, $\vehvel$, a constant acceleration, $\vehacc$ and deceleration, $\vehdec$.
For simplicity we do not consider different drivering behaviours.
We operate with two types of vehicles: cars and trucks. See Table \ref{table.vehicleTypes} for their specifications.
\begin{table}
\centering
\begin{tabular}{|l|l|l|}\hline
		& Car	& Truck \\\hline
Max speed 	& 	& \\\hline
Acceleration 	&	& \\\hline
Deceleration 	&	& \\\hline
\end{tabular}
\caption{Types of vehicles}\label{table.vehicleTypes}
\end{table}

\subsection{Sensors}
A sensor can detect whether a vehicle is located at the sensor or not. Sensors can be induction loops, video cameras or others.

\subsection{Traffic Light}
A traffic light, $\tl$ controls the flow of traffic at a junction (vertex), and is associated with the vertice, $\tlver$ and a set of sensors, $\tlsen$.
Even though we represent a juction as a vertex it has a physical size and geometry, as illustrated in Figure~\ref{fig:Introduction:network}.
To get the distance from a traffic light to a vehicle, we simply calculate the distance to the vertex at the center of the traffic and substract the average distance from the center to the outer egdes of the traffic light. %TODO: Its is too complex?

Hence, we cannot predict the precise distance to the position at a traffic light where the driver must stop.
However, there is already inaccuracies due to blocking cars.
This can for example be observed in Figure \ref{fig:Introduction:network} where vehicle $\veh_1$ drives towards traffic light $\tl_2$. 
He believes that the distance to were he has to stop is the distance $\dist_1$, however, the two blue cars already waiting in line blocks a part of the road.




