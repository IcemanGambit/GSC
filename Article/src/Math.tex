\section{Calculating the Recommended Speed}

We base the recommended speed, $\vel$ for arriving at the next traffic light while it is green, on the definition of velocity.
We use two velocities: one for arriving at the traffic light just as it changes to green and one as it chances to red.
We want to reach the junction just as it changes to green, because this will alow more vehichles to pass the junction afterwards.
We do, however, not want to drive faster than the speed limit.
The recommended speed is therefore somewhere between the speed limit, $\velmax$ and the speed for reaching a red light, $\velslow$:
\begin{eqnarray}
\velrec &=& \frac{\dist}{\tgreen}\label{eq:Math:h}\\
\velslow &=& \frac{\dist}{\tslow}\nonumber\\
\velslow < &\velrec& \leq \velmax\label{eq:Math:recSpeed}
\end{eqnarray}
where
\vspace{-5mm}
\begin{itemize*}
\item $\dist$ is the distance between the vehicle and the traffic light
\item $\tgreen$ is the number of seconds before the traffic light changes to green
\item $\tslow$ is the number of seconds before the traffic light changes to red
\end{itemize*}

Algorithm~\ref{alg.recommendedSpeed} shows the procedure for finding the recommended speed $\velrec$ of Equation~\ref{eq:Math:recSpeed}.
The input of the algorithm is the vehicle, \veh, the route that vehicle is traversing, \route, the map, $\map = (V, E, C)$ and the maximum distance we want to look into the future, \distmax. 
\distmax=2.000 meters as default. %TODO: Check max dist
In line~\ref{alg:recSpeed:cnext}, \cnext is the next connection on the route, \route. If there are no more connections, i.e. junctions, on the route then $\cnext=none$.
The distance function on line~\ref{alg:recSpeed:distance} calculates the eucledian distance from the vehicle to the connection.
In SUMO, one cannot get a distance to the connection, but only to the center point of the associated junction.
We therefore calculate the distance to the center of the junction and substract the average distance from the center to the outer egdes of the junctions geometric object. 
The condition in line~\ref{alg:recSpeed:injunction} checks if the vehicle currently is on a connection, i.e. in a junction, or if distance to the connection is more than the maximum distance. 
If so, we should just recommend the maximum allowed speed. 
As a connection does not have a speed limit, we recommend the speed limit of the previous edge.
Otherwise, the maximum speed is the speed limit of the current edge.
The function \getSpan{\cnext, \ti} in line~\ref{alg:recSpeed:getSpan} read the phase of the connection and the current time and returns a sequence of timestamps where the phase of that connection is green. The format is $\langle(\tstart, \tend), (\tstart, \tend), \dots\rangle$.
The algorithm is not given here.
We initialise the recommended speed and the slowest speed in line~\ref{alg:recSpeed:velrec} and \ref{alg:recSpeed:velslow}, respectively.

The for-loop in line~\ref{alg:recSpeed:loopSpans} to line~\ref{alg:recSpeed:loopSpansEnd} loops through the spans and finds the first span that the vehicle can reach.
First, we calculate the number of seconds before the ligth turns green and red in line~\ref{alg:recSpeed:tg} and~\ref{alg:recSpeed:tg}, respectively, and then calculate the slowest speed at which we can drive in order to reach the first green light in line~\ref{alg:recSpeed:hr}.
In line \ref{alg:recSpeed:continue} we check if this span is reachable and continue the for-loop to the next span if it is not. 
The span is not reachable if the span ends before the current time or if the slowest speed is higher than the maximal speed.
Knowing that this span is reachable, we calculate the recommended speed $\velrec$.
If the light is green right know, i.e. $\tgreen \leq 0$, then we again return the maximum speed in line~\ref{alg:recSpeed:green}. 
Otherwise, we calculate $\velrec$ as in \eqref{eq:Math:h} in line~\ref{alg:recSpeed:h}.
We return the recommended speed in line~\ref{alg:recSpeed:returnh}.
By returning the minimum of \velrec and \velmax we ensure that we do not recommend a speed higher than the speed limit.
Should it not be possible to find a reachable span, we just return the maximum speed in line~\ref{alg:recSpeed:returnmax}.

\begin{algorithm}
\caption{recommendSpeed(\veh, \route, $(V, E, C)$, \distmax)}\label{alg.recommendedSpeed}
\begin{algorithmic}[1]
\State $\cnext= $ next \c in \route\Comment $\cnext=none$ if no junction ahead \label{alg:recSpeed:cnext}
\State $\dist=distance(\veh,\cnext)$. \Comment $d=\infty$ if $\cnext=none$\label{alg:recSpeed:distance}

\If {$\vehpos \in C$ or $d>\distmax$}\State \Comment \veh in a junction or \cnext too far ahead\label{alg:recSpeed:injunction}
\State\Return \espeed s.t. \eend=\vehposstart
\Else
\State \velmax = \vehposspeed
\EndIf

\State $\ti=$ current time
\State $spans = getSpans(\cnext, \ti)$ \Comment where phase is $green$ \label{alg:recSpeed:getSpan}

\State $\velrec = 0$ \Comment Recommended speed \label{alg:recSpeed:velrec}
\State $\velslow = 0$ \Comment Slowest speed for reaching a green light \label{alg:recSpeed:velslow}

\ForAll {$(\tstart, \tend)$ in $spans$}\label{alg:recSpeed:loopSpans}
\State $\tgreen = \tstart - \ti$ \Comment Seconds until \cnextphase is green \label{alg:recSpeed:tg}
\State $\tslow = \tend - \ti$ \Comment Seconds until \cnextphase is red\label{alg:recSpeed:tr}
\State $\velslow = \frac{\dist}{\tslow}$\label{alg:recSpeed:hr}

\If{$\tend\leq \ti$ or $\velslow > \velmax$}\Comment Cannot reach green\label{alg:recSpeed:continue}
\State continue
\EndIf

\If{$\tgreen \leq 0$}\Comment light is green\label{alg:recSpeed:green}
\State $\velrec=\velmax$
\Else
\State $\velrec = \frac{\dist}{\tgreen}$\label{alg:recSpeed:h}
\EndIf
\State\Return $min(\velrec, \velmax)$\label{alg:recSpeed:returnh}
\EndFor\label{alg:recSpeed:loopSpansEnd}

\State\Return $\velmax$\label{alg:recSpeed:returnmax}

\end{algorithmic}
\end{algorithm}