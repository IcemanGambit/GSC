\section{The Recommended Speed}\label{sec:Math}

We base the recommended speed, $\vel$ for arriving at the next traffic light while it is green, on the definition of velocity.
\begin{eqnarray}
\velrec = \frac{\dist}{\ti}\label{eq:Math:velocityDefinition}
\end{eqnarray}
where $\dist$ is the distance to the next traffic light and \ti is the time it has to cross that distance.
This involves some inaccuracies as we do not take the acceleration time into account. %v = v0 + a*t
In order to do this, we would need to integrate over time to get the distance travelled during acceleration.
Such calculations are relatively complex and in many cases unnecessary. 
By recomputing each second, the recommended speed will, however, change each time until a fixed point is reached. 
A driver will generally not detect this unless a green light suddenly cannot be reached, and the recommended speed is dramatically decreased.
This is acceptable, for the sake of simple calculations.
The distance \dist is the distance between the vehicle and the traffic light, not accounting for blocking vehicles.
The means, that the actual distance that the vehicle can drive is different.
See for example Figure~\ref{fig:Introduction:network} where vehicle $vh1$ believes the distance is $\dist_2$ where it is actually $\dist_1$.


The driver should reach the junction just as it changes to green, because this will allow more vehicles to pass the junction afterwards.
However, the driver may not drive faster than the speed limit, $\velmax$.
If it is not possible to reach the next traffic light the first time it is green, then the driver should slow down and wait for the next time it is green.
By setting \ti in Equation~\ref{eq:Math:velocityDefinition} to when the green period ends, \tslow, we find how slow the driver can drive and still cross the junction.
\begin{eqnarray}
\velslow = \frac{\dist}{\tslow}\label{eq:Math:velocitySlow}
\end{eqnarray}
Then, if $\velslow > \velmax$ then the driver should slow down and wait for the next time the connection has a green light.

\subsection{Green Spans}\label{sec:greenSpans}
Algorithm~\ref{alg.getSpans} details how to find the next green periods or spans of a connection, \con at a given timestamp, \ti, assuming the traffic light is pre-timed.
The phase of \con is repeated, to give a sequence of spans that is terminated by an upper time limit, \tmax.
The sequence consists of pairs of timestamps:
\[\langle(\ti_0, \ti_1), (\ti_2, \ti_3), \dots, (\ti_{n-1}, \ti_n)\rangle\]
Each value $\ti_i$ where $0\leq i\cdot 2\leq n$ is a timestamp after \ti when the signal for the connection turns green, and each value $\ti_j$ where $1\leq j\cdot 2\leq n-1$ is a timestamp after $\ti_{j-1}$ and before $\ti_{j+1}$ where the signal turns red.

To find the green spans, we first convert the phase, \phase into a sequence of green spans (line~\ref{alg:getSpans:initGreens} through~\ref{alg:getSpans:GreensEnd}).
The converted phase will be stored in a sequence denoted \greens, which is initialised in line~\ref{alg:getSpans:initGreens}.
The variable, \greenStart in line~\ref{alg:getSpans:initgreenS} is used to store the time when a green span begins, until it can be added to \greens. 
It is initialised to \notset, and will have this value when a time is not stored.
The for-loop in line~\ref{alg:getSpans:GreensFor} loops through the phase.
$f$ denotes the color of the signal and $t$ denotes how many seconds the signal will have this color.
In line \ref{alg:getSpans:GreenBegin}, we check if the signal is green. 
We also check if \greenStart is not set, because a phase can have consecutive green signals, and we only want to store the time of the first one.
If a new green is found, we store its associative time in line~\ref{alg:getSpans:storet}.
The green period ends when the light setting is not green and \greenStart is not \notset (line~\ref{alg:getSpans:GreenEnd}).
If such a case is found, the start and end times are appended to $greens$ in line~\ref{alg:getSpans:addGreens} and \greenStart is set to \notset in line~\ref{alg:getSpans:resetnotset}.
The sequence $greens$ will hence end up containing the periods when the connection has a green signal.
If $\cphase=\langle (green, 20), (yellow, 4), (red, 20), (yellow, 2)\rangle$ then $greens=\langle (0,20)\rangle$.

Converting \greens to the desired spans is done in line~\ref{alg:getSpans:ncirc} through~\ref{alg:getSpans:endWhile}.
\ncirc in line~\ref{alg:getSpans:ncirc} is the number of times that the signal has made a full circulation.
$\tsim-\ti$ is the number of seconds passed since the beginning of the simulation and $\Ccirc{\phase}$ is the circulation time of the phase. 
By dividing these, we get the number of circulations.
\spans is initialised in line~\ref{alg:getSpans:initSpans} and will contain the desired green spans.
$i$ in line~\ref{alg:getSpans:i} represent the extra number of circulations we want to look into the future.
The loops in line~\ref{alg:getSpans:whilebegin} and~\ref{alg:getSpans:forbegin} continuously loop through $greens$ and add spans to \spans.
\tstart is the relative time in seconds when a green span begins and \tend is when the span ends.
With $\ncirc$ being the number of circulations since the beginning of the simulation and $i$ being the number of extra circulations, $\tvar$ in line~\ref{alg:getSpans:tvar} becomes the number of seconds we need to add to the values in $greens$ from the start of the simulation.
In line~\ref{alg:getSpans:ifReachable}, we check if this span ends after the currect time.
If $\tend + \tvar$ is larger than the number of seconds the simulation has run, then the span is added to the sequence, \spans in line~\ref{alg:getSpans:add}.
By adding $\tstart + \tvar$ to the timestamp $\tsim$ we get the timestamp when the signal will be green. Likewise for \tend.
$(\tstart', \tend')$ in line~\ref{alg:getSpans:last} is the last element of the sequence.
The while loop runs until the condition in line~\ref{alg:getSpans:breakCond} is met.
This happens when at least one span has been found and when the last element ends after \tmax seconds after \ti.
Having reached the end of the while loop in line~\ref{alg:getSpans:i++}, we increment $i$, and loop one more circulation into the future in the next loop.

Using the same example as above, $\tmax=100$ and $\tsim=\verb#12:42:13#$ we get that $getSpan(\con, \verb#12:53:49#)$ returns $\langle(\verb#12:53:43#,$ $\verb#12:54:03#),$ $(\verb#12:54:29#,$ $\verb#12:54:49#),$ $(\verb#12:55:15#,$ $\verb#12:55:35#)\rangle$.

\begin{algorithm}
\caption{$getSpans(\con, \ti)$}\label{alg.getSpans}
\begin{algorithmic}[1]
\State $greens = \langle\ \rangle$\label{alg:getSpans:initGreens}
\State $\greenStart = \notset$\label{alg:getSpans:initgreenS}
\For{$(\light, \ti')\in \cphase$}\label{alg:getSpans:GreensFor}
\If{$\light = green$ and $\greenStart=\notset$}\label{alg:getSpans:GreenBegin}
\State $\greenStart = \ti'$\label{alg:getSpans:storet}
\ElsIf{$\light\neq green$ and $\greenStart\neq \notset$}\label{alg:getSpans:GreenEnd}
\State $greens.add((\greenStart, \ti'))$\label{alg:getSpans:addGreens}
\State $\greenStart=\notset$\label{alg:getSpans:resetnotset}
\EndIf
\EndFor\label{alg:getSpans:GreensEnd}

\State \ncirc = $\left \lfloor \frac{\tsim-\ti}{\Ccirc{\cphase}} \right \rfloor$\Comment Number of circulations\label{alg:getSpans:ncirc}
\State $\spans = \langle\ \rangle$\label{alg:getSpans:initSpans}
\State $i = 0$\label{alg:getSpans:i}
\While{true}\label{alg:getSpans:whilebegin}
\For{$(\tstart, \tend)\in greens$}\label{alg:getSpans:forbegin}
\State $\tvar = (\ncirc+i)\cdot \Ccirc{\cphase}$\label{alg:getSpans:tvar}
\If {$\tend+\tvar> \tsim-\ti$}\Comment Reachable span\label{alg:getSpans:ifReachable}
\State $\spans.add((\tstart + \tvar+\tsim, \tend + \tvar+ \tsim))$\label{alg:getSpans:add}
\EndIf
\EndFor
\State $(\tstart',\tend') = \spans.last()$ \Comment Last element\label{alg:getSpans:last}
\If {$\spans.length > 0$ and $\tend' \geq \ti+\tmax$}\label{alg:getSpans:breakCond}
\State break\Comment Last span ends after max time
\EndIf
\State $i=i+1$\label{alg:getSpans:i++}
\EndWhile\label{alg:getSpans:endWhile}
\State\Return \spans
\end{algorithmic}
\end{algorithm}



\subsection{Recommended Speed}
Algorithm~\ref{alg.recommendedSpeed} shows the procedure for finding the recommended speed $\velrec$.
The input of the algorithm is the vehicle, \veh, the route, \route and the map, $\map = (V, E, C)$.

First, we find the current speed limit, \velmax in line~\ref{alg:recSpeed:injunction} through~\ref{alg:recSpeed:injunctionEnd}. %Comment: whould it not be easier to let a connection have a speed limit, or is that just wrong?
If the vehicle currently is on a connection, i.e. in a junction, we use the speed limit of the previous edge.
Otherwise, we use that of the current edge.

Next, we find the distance to the next connection, \cnext on the route, \route in line~\ref{alg:recSpeed:cnext}. 
If there are no more connections, i.e. junctions, on the route then $\cnext=none$.
We find the distance from the vehicle to the next connection in line~\ref{alg:recSpeed:distance} and~\ref{alg:recSpeed:dinfty}.
The distance function on line~\ref{alg:recSpeed:distance} calculates the euclidean distance from the vehicle to the connection.
In SUMO, one cannot get a distance to the connection, but only to the center point of the associated junction.
We therefore calculate the distance to the center of the junction and subtract the average distance from the center to the outer edges of the junctions geometric object. %Comment: Could we not just have added the lengths of the edges? But then again this would not be a solution in the real world
If the connection is $none$, the distance is set to infinite in line~\ref{alg:recSpeed:dinfty}.
If the next connection is too far ahead, we just recommend driving at the speed limit. 
Therefore, we simply return \velmax in line~\ref{alg:recSpeed:dmax} if \dist is larger than the maximum look-ahead distance.

The current time, \ti is set in line~\ref{alg:recSpeed:ti} where $current\_time()$ returns a timestamp.
In line~\ref{alg:recSpeed:getSpan}, we call the function $getSpans(\cnext, \ti)$ to get the sequence of timestamps where the phase is green.
We initialise the recommended speed and the slowest speed in line~\ref{alg:recSpeed:velrec} and \ref{alg:recSpeed:velslow}, respectively.
The for-loop in line~\ref{alg:recSpeed:loopSpans} to line~\ref{alg:recSpeed:loopSpansEnd} loops through the spans and finds the first span that the vehicle can reach.
First, we calculate the number of seconds before the light turns green and red in line~\ref{alg:recSpeed:tg} and~\ref{alg:recSpeed:tr}, respectively, and then calculate the slowest speed at which we can drive in order to reach the first green light in line~\ref{alg:recSpeed:hr}.
In line \ref{alg:recSpeed:continue} we check if this span is reachable and continue the for-loop to the next span if it is not. 
The span is not reachable if the slowest speed is higher than the maximal speed, meaning we would have to driver faster than the speed limit, in order to reach it.

Knowing that this span is reachable, we calculate the recommended speed $\velrec$.
If the light is green right know, i.e. $\tgreen \leq 0$, then \velrec is the speed limit in line~\ref{alg:recSpeed:green}. 
Otherwise, we calculate $\velrec$ in line~\ref{alg:recSpeed:h} as in Equation~\ref{eq:Math:velocityDefinition}.
It is difficult to drive very slow, and we therefore set a lower limit of $\vellower=15km/h$ on the recommended speed in line~\ref{alg:recSpeed:lowerLimit}. 
By returning the minimum of \velrec and \velmax in line~\ref{alg:recSpeed:returnh} we ensure that we do not recommend a speed higher than the speed limit.
It might not be possible to find a reachable span, if the circulation time of the phase is larger than the look-ahead time, \tmax (see Section~\ref{sec:greenSpans}).
If so, we just return the maximum speed in line~\ref{alg:recSpeed:returnmax}.

Assume a speed limit of $\velmax = 16 m/s$ and that the vehicle is $200 m$ from the next connection, \con.
\vspace{-4mm}
\begin{itemize}
\item Assume $getSpan(\con, \verb#12:53:49#)$ returns $\langle(\verb#12:53:39#,$\\$\verb#12:54:09#),$ $(\verb#12:54:45#,$ $\verb#12:55:15#),$ $(\verb#12:55:51#,$ $\verb#12:56:21#)\rangle$, then $\velslow < \velmax$ and Algorithm~\ref{alg.recommendedSpeed} returns \velmax as $\verb#12:53:39#-\verb#12:53:49# \leq 0$.

\item Assume $getSpan(\con, \verb#12:53:49#)$ returns $\langle(\verb#12:53:59#,$ $\verb#12:54:29#),$ $(\verb#12:55:05#,$ $\verb#12:55:35#)\rangle$, then \velslow is $5 m/s$.
The first span is therefore reachable. 
The recommended speed is calculated to $20m/s$ which is reduced to $16 m/s$ in order to stay below the speed limit.

\item Assume $getSpan(\con, \verb#12:53:49#)$ returns $\langle(\verb#12:53:44#,$ $\verb#12:53:59#),$ $(\verb#12:54:39#,$ $\verb#12:54:54#)\rangle$, then \velslow is $20 m/s$ and the first span is not reachable.
Looking at the next span, \velslow becomes $3.1m/s$ which is less than the speed limit and $\velrec = 4 m/s$. 
This is less than the lower limit and the recommended speed becomes 4.1 $m/s$.
\end{itemize}

\begin{algorithm}
\caption{recommendSpeed(\veh, \route, $(V, E, C)$)}\label{alg.recommendedSpeed}
\begin{algorithmic}[1]
\If {$\vehpos \in C$} \Comment \veh in a junction\label{alg:recSpeed:injunction}
\State\velmax= \espeed where \eend=\vehposstart
\Else\ \velmax = \vehposspeed
\EndIf\label{alg:recSpeed:injunctionEnd}

\State $\cnext= $ next \con in \route\Comment $\cnext=none$ if no junction ahead \label{alg:recSpeed:cnext}
\If {$\cnext\neq none$}\ $\dist=distance(\veh,\cnext)$ \label{alg:recSpeed:distance}
\Else\ $\dist = \infty$\label{alg:recSpeed:dinfty}
\EndIf

\If{$d>\distmax$} \Return \velmax\Comment \cnext too far ahead\label{alg:recSpeed:dmax}
\EndIf
\State $\ti=$ current\_time()\label{alg:recSpeed:ti}
\State $spans = getSpans(\cnext, \ti)$ \Comment Green spans of \cnext \label{alg:recSpeed:getSpan}

\State $\velrec = 0$ \Comment Recommended speed \label{alg:recSpeed:velrec}
\State $\velslow = 0$ \Comment Slowest speed for reaching a green light \label{alg:recSpeed:velslow}

\ForAll {$(\tstart, \tend)$ in $spans$}\label{alg:recSpeed:loopSpans}
\State $\tgreen = \tstart - \ti$ \Comment Seconds until \cnextphase is green \label{alg:recSpeed:tg}
\State $\tslow = \tend - \ti$ \Comment Seconds until \cnextphase is red\label{alg:recSpeed:tr}
\State $\velslow = \frac{\dist}{\tslow}$\label{alg:recSpeed:hr}

\State\Comment Span is not reachable
\If{$\velslow > \velmax$} continue\label{alg:recSpeed:continue}
\EndIf

\If{$\tgreen \leq 0$} $\velrec=\velmax$\Comment Light is green\label{alg:recSpeed:green}
\Else\ $\velrec = \frac{\dist}{\tgreen}$\Comment Light is not green\label{alg:recSpeed:h}
\EndIf
\State $\velrec = \max(\velrec, \vellower)$\Comment Set lower limit\label{alg:recSpeed:lowerLimit}
\State\Return $\min(\velrec, \velmax)$\Comment Set upper limit\label{alg:recSpeed:returnh}
\EndFor\label{alg:recSpeed:loopSpansEnd}

\State\Return $\velmax$\Comment No reachable green span\label{alg:recSpeed:returnmax}
\end{algorithmic}
\end{algorithm}

\subsection{Complexity and Communication Analysis}
Understanding the complexity and communication costs of \tech is important in order to know whether for example a smartphone implementation is feasible.

Algorithm~\ref{alg.getSpans} has two parts: converting the phase, \phase to green spans and expanding these spans.
Remembering that $|\phase|$ is the number of elements in the phase, then converting the phase has a complexity of $O(|\phase|)$.
Expanding the phase will be dependent on both $|\phase|$ and the number of extra circulations looked into the future. 
The number of extra circulations are 
\[\fcirc = \cfrac{\tmax}{\Ccirc{\phase}}\]
as the returned sequence is terminated when the last element ends after $\tmax$ seconds.
Hence the complexity of Algorithm~\ref{alg.getSpans} is  $O(|\phase|+ |\phase|\cdot \fcirc)$ which can be reduced to $O(|\phase|\cdot \fcirc)$.

The complexity of Algorithm~\ref{alg.recommendedSpeed} is fourth fold: checking if the vehicle is on a connection, finding the next connection on the route, getting the green spans and finding the speed.
Checking if the vehicle is on a connection will be $O(|C|)$ where $|C|$ denotes the number of connections in the network.
However, by setting a type on edges and connections, this can be reduced to a simple lookup and hence $O(1)$.
Using these types, finding the next connection will have a complexity of $O(|\route|)$.
This can be precomputed such that the complexity also becomes $O(1)$.
The route is the sequence of edges and connections to be visited.
By keeping a reduced route with only the connections and the speed limit of the previous edge, and popping this sequence as connections are visited, the complexity will be $O(1)$.
The complexity of getting the green spans is, as stated, $O(|\phase|\cdot \fcirc)$.
Finding the speed is also $O(|\phase|\cdot \fcirc)$ as it loops through the spans, and the content of the loop only has constant complexity.
In total, the complexity of Algorithm~\ref{alg.recommendedSpeed} is $O(|\phase|\cdot \fcirc)$.

The value \fcirc is how many extra circulations of the phase that is needed to get the spans. 
We predict that it often will not be necessary to look more than $2\cdot |\phase|$ in to the future.
In the simulations, \tmax is set to 400 $s$, about 6 and a half minute, to ensure a span is always found.
The circulation time, $\Ccirc{\phase}$ is rarely more than 120 seconds, so $\fcirc$ will be a small number.
Additionally, $|\phase|$ will in most case be short and cannot be higher than than the circulation time.
We can therefore conclude that the complexity of recommending a speed is low.

The communication cost is also low.
We propose a anonymous, pull strategy where a device asks a server for the spans of a traffic light given its identifier.
We estimate one pull request per second should be sufficient.
This is the same frequency used in the the following simulations.
The pull request will only contain an integer identifying the connection of the next traffic light on the route.
The response will as explained above be proportional to the length of the green spans, and hence $O(|\phase|\cdot \fcirc)$.
Each span entry will be a pair of integers.
The communication cost is thereby also small and asymptotic to $O(|\phase|\cdot \fcirc)\cdot 2\cdot size\_of(integer)$. %, where $si$ is the size of an integer ($4$ bytes). 
Simple experiments show that $|\phase|\cdot \fcirc$ is rarely more than ten. 
This gives an estimated communication cost of 4 $bytes/s$ in the request and 80 $bytes/s$ in the response, not including communication headers and time synchronisation.
The total communication cost can therefore be estimated to 84 $bytes/s$, about 5 kilobytes per minute.





